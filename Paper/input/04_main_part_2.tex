\chapter{Skill Implementation}%: \inote{Kosten-/Terminabfrage}}
\label{maintwo}

\section{Setup}


step by step code analysis

\url{https://github.com/alexa/alexa-skills-kit-sdk-for-nodejs/blob/master/Readme.md}
\subsection{requirements}

for audio

json format https://stackoverflow.com/questions/41776014/how-to-correctly-specify-ssml-in-an-alexa-skill-lambda-function



HTTP(S) endpoint:
\url{http://newsreel-edu.aot.tu-berlin.de/solr/#/d115}


references:

\url{https://developer.amazon.com/docs/custom-skills/speech-synthesis-markup-language-ssml-reference.html#audio}



blogs:
\url{https://developer.amazon.com/post/Tx3FXYSTHS579WO/Announcing-New-Alexa-Skills-Kit-ASK-Features-SSML-Audio-Tags-and-Developer-Porta}

With the JSON we have, we try to maximise its use, but we need to consider that with starting a Skill, it is better to start from scratch sometimes and as we go, we see what is the relevant information that we can integrate into our Skill.

We analyze the output of Virtueller Bürgerassistent and realize it makes less sense to start with that, so we refer to the underlying endpoint. Since this is the Solr Server that delivers JSON objects and is more realistic to maneuver, we analyse the queries we can get from there. and try to fit it into our interaction model.

Then we move on to the interaction model. On paper, we draft the use-cases to determine what are our intents first to group these into fewer intents than the services we have. Given that there are many public services related to a similar case, we design our Skill to make it possible to group them into less intents. The advantage of this is that it allows us to at least have fewer 'ServiceIntents' than the ever growing number of services, which means that we want to also be able to track new services as they get inserted into the catalogue to be able to either map them to old intents by extending these, or to introduce new intents if this does not work.



\todo{ moved from intro\\


our now more than \inote{4}
scenarios:
\begin{itemize}
	\item a general scenario of predefined categories related to any service \tocite{hier schon erwähnen? \inote{(prerequisites, costs, documents, ...)}}
	\item special application on \textbf{Residence Registration} - ``Anmeldung einer Wohnung''
	\item special application on \textbf{Applying for a Residence Permit} - mutliple services related to ``Aufenthaltstitel''
	\item ``car registration'' -  special application on \textbf{Car Registration''} - multiple services  related to ``Kraftfahrzeug (KFZ)''
	%	\item[scenario 4] \inote{yet to be decided or if to implement}
\end{itemize}}


\section{step-by-step}

start with interaction model
then use builder
then fulfill functions


steps to git commit
deploy
clone
change in browser
change offline

\section{Code Analysis}

from \url{https://github.com/alexa/alexa-skills-kit-sdk-for-nodejs/blob/master/Readme.md}
The difference between :ask/listen and :tell/speak is that after a :tell/speak action, the session is ended without waiting for the user to provide more input. We will compare the two ways using response or using responseBuilder to create the response object in next section.




jsons sent/received in appendix
\ref{jsonFromAlexa}




problems with countrylist

\section{Deployment}

\section{Approval and Publishing}

\section{Debugging / Troubleshooting}

\section{Documentation Updates}
https://www.gitbook.com/join/tu-berlin/-LAZYgGmoeaEZGb8cYg2
https://tu-berlin.gitbook.io/alexa/

% Implementation as Facebook Messenger Bot / Google Action
\textcolor{magenta}{
- as an example for text\\
- implementing the answer suggestions as buttons\\
- passing data to the Bürgeramt terminseite\\
https://console.dialogflow.com/api-client/ \\
https://console.actions.google.com
}


%Solr is not secure (ein kleines thing in future work)
Solr has to be secure
\url{https://docs.aws.amazon.com/AmazonS3/latest/dev/WebsiteEndpoints.html#WebsiteRestEndpointDiff}


Hey:

do this set up as prerequisites first
as a deployment script

- install aws cli
- install ask cli
- set up a profile on AWS
- ask init etc.

https://docs.aws.amazon.com/cli/latest/userguide/cli-chap-getting-started.html

There is no AWS credential setup yet, do you want to continue the initialization?

https://developer.amazon.com/docs/smapi/quick-start-alexa-skills-kit-command-line-interface.html



Manage IAM roles (User, groups, rights, policies etc.)


all logs go onto cloudwatch
including when the request is not right etc...can be helpful to know what users want, beyond testing





amazon does not disclose avaliable values for their slot types (lists), but they give you examples here
%https://developer.amazon.com/docs/custom-skills/slot-type-reference.html#h2_extend_types


===

finally, we had to resolve to the information publicly available to tailor custom scenarios



 ===
once you upload the new lambda, the old one is gone, unless you version it like this:
https://docs.aws.amazon.com/lambda/latest/dg/versioning-aliases.html



====

first i got screwed over with this Big nerd ranch then this happened: alexa-skill module
- so don't use this tutorial, although very helpful for a beginner and makes you understand what happens under the hood, it has been abstracted in many other function and the logic is no longer the same. - which explains why no one starred it

https://github.com/matt-kruse/alexa-app

https://www.bignerdranch.com/blog/developing-alexa-skills-locally-with-nodejs-deploying-your-skill-to-staging/



\todo{
	OnLaunch\\
	IntentHandler\\
	intent is triggered by utterence\\
	account verlinkungen etc\\
}

Prob with outdated tutorials