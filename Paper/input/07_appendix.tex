\chapter*{Appendices}
\label{appendices}
%DELETEME: everything that does not fit into your work, e.g. a 5 page table that breaks the reading flow, should be placed here


\addcontentsline{toc}{section}{Annex}
%excursus
\section*{Annex: On Etymology}
\label{etymology}

Product naming is a branding problem which companies deal with differently. In the case of Apple, with exception of the `Apple Watch' and on their consumer products line, they opt for a clever version using the letter `i' in front of simple common words, like iBook, iPad, iPhone. though this naming convention originally arose from `i' to stand for `interactive' celebrating the early adoption of the internet on the Macintosh product line, it has turned into a naming breakthrough since it had two effects: 1) it gives the user of Apple software or hardware product a feeling of personalisation for the letter `i' would make something sound as belonging to oneself, namely `` `I', as a person'' as a prefix for the name of the object create the name of the product, and 2) it is a marketing strategy to associate all products and services carrying the prefix `i' with Apple without the need to name the company's name. This applies to products and services that came to life even long after people's adoption of the internet and the widespread use of Apple products, like with iCloud as a web service. Another example is with professional software products like AutoCAD combining the company's name (AutoDesk) with the product's functionality  (Computer Aided Design). Companies like Adobe clearly prefer to name their products and services with their brand's full name included, as the case with Adobe Photoshop, Adobe Flash (formely know as MacroMedia Flash). Similarly, as the youngest of them, Amazon adopts a combined approach by stating the full name of the company on its consumer line like with `Amazon Prime', `Amazon Alexa' and a mix between `AWS' and `Amazon' like with `Amazon SageMaker', `AWS Firewall Manager' etc. on its developer B2B line.  


\todo{did something go missing here? check git!}
%amazon transcri ?


\section*{Difference Between Lex and Alexa Skills}
\label{lexAlexa}  

\todo{cite? indent? copied text}
Amazon Lex is a service for building conversational interfaces using voice and text. Powered by the same conversational engine as Alexa, Amazon Lex provides high quality speech recognition and language understanding capabilities, enabling addition of sophisticated, natural language \'chatbots\' to new and existing applications. Amazon Lex reduces multi-platform development effort, allowing you to easily publish your speech or text chatbots to mobile devices and multiple chat services, like Facebook Messenger, Slack, Kik, or Twilio SMS. Native interoperability with AWS Lambda, AWS MobileHub and Amazon CloudWatch and easy integration with many other services on the AWS platform including Amazon Cognito, and Amazon DynamoDB makes bot development effortless.
For more information: \url{https://stackoverflow.com/questions/42982159/differences-between-using-lex-and-alexa#URL}\\
\url{https://aws.amazon.com/about-aws/whats-new/2017/09/export-your-amazon-lex-chatbot-to-the-alexa-skills-kit/}\\
\url{https://aws.amazon.com/lex/faqs/}



%###################################################################################
%###################### Appendix A          ########################################
%###################################################################################
%uncomment, if desired
%\newpage
\addcontentsline{toc}{section}{Appendix A: Abbreviations}
\section*{Appendix A: Abbreviations}
\begin{flushleft}
\begin{tabular}{ll}
%\textbf{AES}	&	Advanced Encryption Standard (Symmetrisches Verschlüsselungsverfahren)\\
%\textbf{ASCII}&	American Standard Code for Information Interchange (Computer-Textstandard)\\
%\textbf{dpi}	&	dots per intch (Punkte pro Zoll; Maß für Auflösung von Bilddateien)\\
%\textbf{HTML}	&	Hypertext Markup Language (Textbasierte Webbeschreibungssprache)\\
%%\textbf{JAP}	&	Java Anon Proxy\\
%\textbf{JPEG}	&	Joint Photographic Experts Group (Graphic format)\\
%\textbf{JPG}	&	Joint Photographic Experts Group (Graphic format; short form)\\
%\textbf{LED}	&	Light Emitting Diode (lichtemittierende Diode)\\
%\textbf{LSB}	&	Least Significant Bit\\
%\textbf{MD5}	& Message Digest (Kryptographisches Fingerabdruckverfahren)\\
%\textbf{MPEG}	&	Moving Picture Experts Group (Video- einschließlich Audiokompression)\\
%\textbf{MP3}	&	MPEG-1 Audio Layer 3 (Audiokompressionformat)\\
%\textbf{PACS}	&	Picture Archiving and Communication Systems\\
%\textbf{PNG}	&	Portable Network Graphics (Grafikformat)\\
%\textbf{RSA}	&	Rivest, Shamir, Adleman (asymmetrisches Verschlüsselungsverfahren)\\
%\textbf{SHA1}	&	Security Hash Algorithm (Kryptographisches Fingerabdruckverfahren)\\
%\textbf{WAV}	&	Waveform Audio Format (Audiokompressionsformat von Microsoft)\\

\textbf{AWS}	&	Amazon Web Serivces\\
\textbf{ASK}	&	Alexa Skills Kit\\
\textbf{AVS}	&	Alexa Voice Service\\
\textbf{ARN}	&	Amazon Resource Name\\
\textbf{MVP}	&	Minimum Viable Product\\
\textbf{CLI}	&	Command-Line Interface\\
\textbf{MVC}	&	Model-View-Controller\\
\textbf{CIA}	&	Competitive Innovation Advantage\\
%\textbf{abk}	&	erklärung\\
%\textbf{abk}	&	erklärung\\
%\textbf{abk}	&	erklärung\\
\textbf{AI}		&	Artificial Intelligence\\
\textbf{NLP}	&	Natural Language Processing\\
\textbf{ML}		&	Machine Learning\\
\textbf{GUI}	&	Graphical User Interface\\
\textbf{VUI}	&	Voice User Interface\\
\textbf{ACID}	&	Atomicity, Identity, .. criteria\\

\textbf{appx}	&	approximately\\
\end{tabular}
\end{flushleft}

%###################################################################################
%###################### Appendix B          ########################################
%###################################################################################
\newpage
%change this
\addcontentsline{toc}{section}{Appendix B: Glossary}
\section*{Appendix B: Glossary}
%%remove this
%\input{__help/latex_hinweise}

\begin{flushleft}
\begin{tabular}{ll}
\textbf{Intent}	&	erklärung\\
\textbf{Slot}	&	erklärung\\
\\
\textbf{Utterance}	&	erklärung\\
\textbf{Alexa}	&	erklärung\\
\textbf{Alexa Skill}	&	erklärung\\
\\

\textbf{Alexa Skills Kit}	&	erklärung\\
\textbf{Amazon Developer Console}	&	erklärung\\
\textbf{Lambda, AWS Lambda}	&	erklärung\\
\textbf{Lambda Function}	&	erklärung\\
\textbf{Lex, Amazon Lex}	&	erklärung\\
\textbf{Polly, Amazon Polly}	&	erklärung\\
\textbf{Amazon Transcribe}	&	erklärung\\
\textbf{ElasticSearch}	&	erklärung\\
\textbf{Node.js}	&	Framework built on top of JavaScript\\
\textbf{Interaction Model}	&	erklärung\\
\textbf{Service}	&	bot, AWS, Berlin.de\\
\textbf{app}	&	Mobile Application\\
\\
https://docs.aws.amazon.com/general/latest/gr/glos-chap.html
\\
\textbf{Application ID}	&	erklärung\\
\textbf{Skill ID}	&	erklärung\\
\textbf{Bot}	&	Unless otherwise mentioned, yeb2a Chatbot\\
\textbf{Hitlist}	&	erklärung\\
\textbf{Voice-first}	& an always-on, intelligent piece of hardware, where the primary interface is voice, both input and output.\\
\textbf{B2B}	&	business-to-business\\
\textbf{B2C}	&	business-to-consumer\\
\textbf{Inc.}	&	incorporation\\
%\textbf{abk}	&	erklärung\\
%\textbf{abk}	&	erklärung\\
%\textbf{abk}	&	erklärung\\
%\textbf{abk}	&	erklärung\\
%\textbf{abk}	&	erklärung\\


\end{tabular}
\end{flushleft}
