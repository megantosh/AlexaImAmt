\chapter*{Appendix} %{Appendices}
\label{appendices}
%DELETEME: everything that does not fit into your work, e.g. a 5 page table that breaks the reading flow, should be placed here

\todo{check bibstyle - check all relevant docs in Lit review folder are included}

\addcontentsline{toc}{section}{Product Branding Etymology}
%excursus
\section*{On Etymology of Product Names}
\label{etymology}

Product naming is a branding problem which companies deal with differently. In the case of Apple, with exception of the `Apple Watch' and on their consumer products line, they opt for a clever version using the letter `i' in front of simple common words, like iBook, iPad, iPhone. though this naming convention originally arose from `i' to stand for `interactive' celebrating the early adoption of the internet on the Macintosh product line, it has turned into a naming breakthrough since it had two effects: 1) it gives the user of Apple software or hardware product a feeling of personalisation for the letter `i' would make something sound as belonging to oneself, namely `` `I', as a person'' as a prefix for the name of the object create the name of the product, and 2) it is a marketing strategy to associate all products and services carrying the prefix `i' with Apple without the need to name the company's name. This applies to products and services that came to life even long after people's adoption of the internet and the widespread use of Apple products, like with iCloud as a web service. Another example is with professional software products like AutoCAD combining the company's name (AutoDesk) with the product's functionality  (Computer Aided Design). Companies like Adobe clearly prefer to name their products and services with their brand's full name included, as the case with Adobe Photoshop, Adobe Flash (formely know as MacroMedia Flash). Similarly, as the youngest of them, Amazon adopts a combined approach by stating the full name of the company on its consumer line like with `Amazon Prime', `Amazon Alexa' and a mix between `AWS' and `Amazon' like with `Amazon SageMaker', `AWS Firewall Manager' etc. on its developer B2B line.  


%\todo{did something go missing here? check git!}
%amazon transcri ?

\addcontentsline{toc}{section}{Lex vs. Alexa Skills}
\section*{Difference Between Lex and Alexa Skills}
\label{lexAlexa}  

%\todo{cite? indent? copied text}
Amazon Lex is a service for building conversational interfaces using voice and text. Powered by the same conversational engine as Alexa, Amazon Lex provides high quality speech recognition and language understanding capabilities, enabling addition of sophisticated, natural language \'chatbots\' to new and existing applications. Amazon Lex reduces multi-platform development effort, allowing you to easily publish your speech or text chatbots to mobile devices and multiple chat services, like Facebook Messenger, Slack, Kik, or Twilio SMS. Native interoperability with AWS Lambda, AWS MobileHub and Amazon CloudWatch and easy integration with many other services on the AWS platform including Amazon Cognito, and Amazon DynamoDB makes bot development effortless. \footnote{
More information: \url{https://stackoverflow.com/questions/42982159/differences-between-using-lex-and-alexa\#URL}\\
\url{https://aws.amazon.com/about-aws/whats-new/2017/09/export-your-amazon-lex-chatbot-to-the-alexa-skills-kit/}\\
\url{https://aws.amazon.com/lex/faqs/}
}









%%%%%%%%%%%%%%%%%%%%%%%%%%%%%%%%%%%%%%%%%%%%%%%%%%%%%%%%%%%%%%%%%%%%%%%%%%%%%
%%%%%%%%%%%%%%%%%%%%%%%%%%%%%%%%%%%%%%%%%%%%%%%%%%%%%%%%%%%%%%%%%%%%%%%%%%%%%
\clearpage

\section*{Documentation (Base) URLs}

\todo{use a base url to avoid extensive text in footnotes with href such that a click would still redirect right\\ -kbase \\ -Alexa blogs\\
\\
AWS Glossary\\ \t{a\t{ws}}\href{www.apple.com}{\lstinline|/doing-something\#hej|}
\url{https://docs.aws.amazon.com/general/latest/gr/glos-chap.html}

}



\subsection*{Amazon Web Services - \t{a\t{ws}}}


\begin{flushleft}
	\begin{tabularx}{\textwidth}{lX}




AWS Glossary &
\url{https://docs.aws.amazon.com/general/latest/gr/glos-chap.html}\\

\end{tabularx}
\end{flushleft}
	


\subsection*{Alexa Skills Kit - \t{a\t{sk}}}
\begin{flushleft}
	\begin{tabularx}{\textwidth}{l X}

Documentation &
\url{https://developer.amazon.com/docs/custom-skills/} \\


\end{tabularx}
\end{flushleft}

\subsection*{GitHub Repositories - \t{GH}}


\begin{flushleft}
	\begin{tabularx}{\textwidth}{lX}
		
		
A Skill Template with API Calls: & 
\url{https://github.com/skilltemplates/api-starter-alexa}\\

Node.js code &
\url{https://github.com/alexa/alexa-skills-kit-sdk-for-nodejs}\\

Building Responses in Node.js & 
\url{https://github.com/alexa/alexa-skills-kit-sdk-for-nodejs/wiki/Response-Building}\\


Alexa Cookbook (maintained by Amazon Team) & 
\url{https://github.com/alexa/alexa-cookbook/}\\

Handling Responses (Multi-turn conversation) & 
\url{https://github.com/alexa/alexa-cookbook/tree/master/handling-responses}\\
		
SSML Audio & 
\url{https://github.com/alexa/alexa-cookbook/tree/0afbfcb8ddcd911e83216b4528b20f3a1628c99f/handling-responses/ssml-audio}\\
		
	\end{tabularx}
\end{flushleft}



%\t{a\t{sk}}\href{   }{\lstinline|   |}

%find \url{https://developer.amazon.com/docs/custom-skills
% replace with \t{a\t{sk}}\href{\url{https://developer.amazon.com/docs/custom-skills}{\lstinline|   

%then |}


%%%%%%%%%%%%%%%%%%%%%%%%%%%%%%%%%%%%%%%%%%%%%%%%%%%%%%%%%%%%%%%%%%%%%%%%%%%%%
\clearpage

\addcontentsline{toc}{section}{JSON Query code structure}
\section*{Outtakes from a JSON query node}
\label{query:dl}

HTML Tags inside Text
\begin{minted}[tabsize=2, bgcolor=bgkolor, breaklines, fontsize=\footnotesize]{json}

{ "d115Description": "Eine Fiktionsbescheinigung wird ausgestellt, wenn über einen beantragten Aufenthaltstitel noch nicht entschieden werden kann, z. B. weil<br />\n<br />\n<ul class=\"list\"><li> ... Ausländerakte ... liegt,</li> ...
\end{minted}

Ambiguous words that can be a plus for the Virtual Citizen Assistant but become irrelevant or confusing for the Skill (e.g. \mintinline{text}{Fiktionsbescheinigung})
\begin{minted}[tabsize=2, bgcolor=bgkolor, breaklines, fontsize=\footnotesize]{json}
{... "d115Synonym": [
"Aufenthaltserlaubnis","Aufenthaltstitel","vorläufig",
"Visum","Fiktionsbescheinigung"] ...
\end{minted}

Inconsistent or unconventional text encoding (e.g. \mintinline{text}{::: false :::} )
\begin{minted}[tabsize=2, bgcolor=bgkolor, breaklines, fontsize=\footnotesize]{json}
{... "d115InfoLaw": ["§ 81 Aufenthaltsgesetz - AufenthG  ::: false ::: http://www.geset...net.de/aufenthg_2004/__81.html"] ...
\end{minted}

Text gets lost in translation from reading to speaking or is hard to read out loud
\begin{minted}[tabsize=2, bgcolor=bgkolor, breaklines, fontsize=\footnotesize]{json}
{... "d115Prerequisites": [ "Persönliche Vorsprache ist erforderlich ::: false ::: ",
"Rechtmäßiger Aufenthalt mit oder ohne Aufenthaltstitel ::: Zum Zeitpunkt des Antrags muss die Antragstellerin oder der Antragsteller entweder<br />\n<br />\n<ul class=\"list\"><li>einen gültigen Aufenthaltstitel (Aufenthaltserlaubnis oder nationales Visum für längerfristige Aufenthalte - Kategorie D - ) [...]",
"Antrag auf Aufenthaltstitel ::: Eine Fiktionsbescheinigung wird nur dann ausgestellt, wenn [...] ::: ",
"Hauptwohnsitz in Berlin ::: false ::: "] ...
\end{minted}

\begin{minted}[tabsize=2, bgcolor=bgkolor, breaklines, fontsize=\footnotesize]{json}
{... "d115Requirements": [ "Bisheriger Aufenthaltstitel ::: Soweit vorhanden, ist der bisherige Aufenthaltstitel mitzubringen, z.B. der elektronische Aufenthaltstitel (eAT). ::: ",
"Nachweis über den Hauptwohnsitz in Berlin ::: <ul class=\"list\"><li>Bescheinigung über die Anmeldung der Wohnung (Meldebestätigung) <strong>oder</strong></li><li>Mietvertrag und Einzugsbestätigung des Vermieters</li></ul>Mehr zum Thema im Abschnitt „Weiterführende Informationen“ ::: ",
"..."] ...
\end{minted}

Not a single readable item price
\begin{minted}[tabsize=2, bgcolor=bgkolor, breaklines, fontsize=\footnotesize]{json}
{..."d115Fees": "Ab dem 01.09.2017:<br />\n<br />\n<ul class=\"list\"><li>Für Erwachsene: 13,00 Euro</li><li>Für Minderjährige: 6,50 Euro</li></ul>Gebührenfrei für<br />\n<ul class=\"list\"><li>Türkische Staatsangehörige</li><li>Asylberechtigte</li><li>
Ausländer, die im Bundesgebiet die Rechtsstellung ausländischer Flüchtlinge oder subsidiär Schutzberechtigter genießen</li><li>Resettlement-
Flüchtlinge im Sinne von § 23 Abs. 4 S. 1 AufenthG</li></ul>" ...
\end{minted}

\begin{minted}[tabsize=2, bgcolor=bgkolor, breaklines, fontsize=\footnotesize]{json}
{..."d115ProcessTime": "Die Fiktionsbescheinigung wird bei Vorsprache ausgestellt." ...
\end{minted}

%
%"d115AppointmentLink": "",
%"d115ServiceLocations": [
%"121885",
%"327437",
%"327805"
%],
%"d115ServiceLocationsJson": [
%"{\"location\":\"327437\",\"url\":\"https://service.berlin.de/dienstleistung/326233/standort/327437/\",\"appointment\":{\"link\":\"\",\"slots\":\"0\",\"external\":\"false\",\"multiple\":\"0\",\"allowed\":\"false\"},\"hint\":\"\"}",
%
%// [two more entries]
%],
%"d115ServiceResponsibilityAll": false,
%"d115OnlineProcessingLink": "",
%"leikaId": 99010008012000,
%"leikaName": "Fiktionsbescheinigung Ausstellung",
%"leikaGruppe": "Aufenthaltstitel",
%"leikaKennung": "Fiktionsbescheinigung",
%"leikaVerrichtung": "Ausstellung",
%"leikaVerrichtungDetail": "",
%"ssdsName": [ "Fiktionsbescheinigung Ausstellung", "Fiktionsbescheinigung ", "Fiktionsbescheinigung"
%],
%"ssdsLemma": [
%"vorläufig", "aufenthaltstitel",
%"ausstellung", "visum", "aufenthaltserlaubnis", "fiktionsbescheinigung"
%],
%"ssdsLongName": "Ausstellung Aufenthaltstitel Fiktionsbescheinigung",
%"ssdsGruppe": [
%"Aufenthaltstitel"
%],
%"ssdsGruppeDict": [
%"Aufenthaltstitel"
%],
%"ssdsKennung": [
%"Fiktionsbescheinigung"
%],
%"ssdsKennungDict": [
%"Fiktionsbescheinigung"
%],
%"ssdsVerrichtung": [
%"Ausstellung"
%],
%"ssdsVerrichtungDict": [
%"Ausstellung"
%],
%"ssdsSynonym": [
%"Aufenthaltserlaubnis", "Aufenthaltstitel", "vorläufig", "Visum", "Fiktionsbescheinigung"],
%"ssdsSynonymDict": [
%"Aufenthaltserlaubnis", "Aufenthaltstitel", "vorläufig", "Visum", "Fiktionsbescheinigung"],
%"ssdsManualKeywords": [
%"Visa"
%],
%"_version_": 1598406083041296400
%},
%\end{minted}












%%%%%%%%%%%%%%%%%%%%%%%%%%%%%%%%%%%%%%%%%%%%%%%%%%%%%%%%%%%%%%%%%%%%%%%%%%%%%
\clearpage

%\todo{check tcolorbox and mited manuals for headings etc}
%\begin{tcblisting}{<tcb options>,
%		minted language=<language>,
%		minted style=<style>,
%		minted options={<option list>} }
%	<code>
%\end{tcblisting}




%\todo{3ashan beyetghayarou bsr3a}
% DO NOT INDENT OR PLAY AROUND HERE LEST THE CODE ZERSCHIESST SICH
\begin{figure}[H]
	\caption[HTTPS Request Body Syntax]{Request Body (26.04.2018) }
	\label{jsonFromAlexa}

\begin{minted}[linenos,tabsize=2, bgcolor=bgkolor, breaklines, fontsize=\footnotesize]{json}
{"version": "1.0",
 "session": {
  "new": true,
  "sessionId": "amzn1.echo-api.session.[unique-value]",
  "application": {
   "applicationId": "amzn1.ask.skill.[unique-value]"},
  "attributes": {
   "key": "string value"},
  "user": {
   "userId": "amzn1.ask.account.[unique-value]",
   "accessToken": "Atza|AAAAAAAA...",
   "permissions": {
   "consentToken": "ZZZZZZZ..."}}},
 "context": {
 "System": {
  "device": {
   "deviceId": "string",
   "supportedInterfaces": {
    "AudioPlayer": {}}},
  "application": {
   "applicationId": "amzn1.ask.skill.[unique-value]"},
  "user": {
   "userId": "amzn1.ask.account.[unique-value]",
   "accessToken": "Atza|AAAAAAAA...",
   "permissions": {
    "consentToken": "ZZZZZZZ..."}},
  "apiEndpoint": "https://api.amazonalexa.com",
  "apiAccessToken": "AxThk..."},
 "AudioPlayer": {
  "playerActivity": "PLAYING",
  "token": "audioplayer-token",
  "offsetInMilliseconds": 0}},
 "request": {}}
	
	\end{minted}
\end{figure}

Can be used to generate tests in Lambda.\\
source: \textsc{ASK Documentation} \footnote{\t{a\t{sk}}\href{https://developer.amazon.com/docs/custom-skills}{\lstinline|/request-and-response-json-reference.html\#request-body-syntax|}}




%###################################################################################
%###################### Appendix A          ########################################
%###################################################################################
%uncomment, if desired
\newpage
\addcontentsline{toc}{section}{Abbreviations}
\section*{List of Acronyms and Abbreviations }
\begin{flushleft}
\begin{tabular}{ll}
%\textbf{AES}	&	Advanced Encryption Standard (Symmetrisches Verschlüsselungsverfahren)\\

%\textbf{dpi}	&	dots per intch (Punkte pro Zoll; Maß für Auflösung von Bilddateien)\\
%\textbf{HTML}	&	Hypertext Markup Language (Textbasierte Webbeschreibungssprache)\\
%%\textbf{JAP}	&	Java Anon Proxy\\
%\textbf{JPEG}	&	Joint Photographic Experts Group (Graphic format)\\
%\textbf{JPG}	&	Joint Photographic Experts Group (Graphic format; short form)\\
%\textbf{LED}	&	Light Emitting Diode (lichtemittierende Diode)\\
%\textbf{LSB}	&	Least Significant Bit\\
%\textbf{MD5}	& Message Digest (Kryptographisches Fingerabdruckverfahren)\\
%\textbf{MPEG}	&	Moving Picture Experts Group (Video- einschließlich Audiokompression)\\
%\textbf{MP3}	&	MPEG-1 Audio Layer 3 (Audiokompressionformat)\\
%\textbf{PACS}	&	Picture Archiving and Communication Systems\\
%\textbf{PNG}	&	Portable Network Graphics (Grafikformat)\\
%\textbf{RSA}	&	Rivest, Shamir, Adleman (asymmetrisches Verschlüsselungsverfahren)\\
%\textbf{SHA1}	&	Security Hash Algorithm (Kryptographisches Fingerabdruckverfahren)\\
%\textbf{WAV}	&	Waveform Audio Format (Audiokompressionsformat von Microsoft)\\


\textbf{ACID}	&	Atomicity, Consistency, Isolation, Durability\\ %still needed?
\textbf{AI}		&	Artificial Intelligence\\
\textbf{ARN}	&	Amazon Resource Name\\
\textbf{ASCII}&	American Standard Code for Information Interchange\\% (Computer-Textstandard)\\
\textbf{ASK}	&	Alexa Skills Kit\\
\textbf{AVS}	&	Alexa Voice Service\\
\textbf{AWS}	&	Amazon Web Serivces\\

\textbf{B2B}	&	business-to-business\\
\textbf{B2C}	&	business-to-consumer\\

\textbf{CEO}	&	Chief Executive Officer\\
\textbf{CIA}	&	Competitive Innovation Advantage\\
\textbf{CLI}	&	Command-Line Interface\\
\textbf{CSS}	&	Cascading Style Sheets\\

\textbf{DCG}	&	Discounted Cumulative Gain\\

\textbf{FAQ}	&	Frequently Asked Question\\

\textbf{GB}		&	Gigabyte\\
\textbf{GUI}	&	Graphical User Interface\\

\textbf{HCI}	&	Human Capital Index\\
\textbf{HTML}	&	Hyper-Text Markup Language\\

\textbf{IoT}	&	Internet of Things\\
\textbf{IVR}	&	Interactive Voice Response\\

\textbf{JSON}	&	JavaScript Object Notation\\

\textbf{kbps}	&	kilobits per second\\

\textbf{LeiKa}	&	Leistungskatalog der öffentlichen Verwaltung \\
& (public administration service catalogue)\\


\textbf{ML}		&	Machine Learning\\
\textbf{MP3}	&	Moving Picture Experts Group Layer-3 Audio (MPEG-3)\\
\textbf{MVC}	&	Model-View-Controller\\
\textbf{MVP}	&	Minimum Viable Product\\

\textbf{NLP}	&	Natural Language Processing\\
\textbf{NLU}	&	Natural Language Understanding\\
\textbf{NPM}	&	Node Package Manager\\

\textbf{OS}		&	Operating System\\
\textbf{OSI}	&	Online Services Index\\
\textbf{ROI}	&	Return on Investment\\
\textbf{SSML}	&	Speech Synthesis Markup Language\\

\textbf{TF-IDF}	&	Term Frequency - Inverted Document Frequency\\
\textbf{TII}	&	Telecommunication Infrastructure Index\\

\textbf{UN}		&	United Nations\\

\textbf{VUI}	&	Voice User Interface\\


%\textbf{abk}	&	erklärung\\
%\textbf{abk}	&	erklärung\\
%\textbf{abk}	&	erklärung\\
%\textbf{abk}	&	erklärung\\
%\textbf{abk}	&	erklärung\\


\end{tabular}
\end{flushleft}


%next page
\begin{flushleft}
	\begin{tabular}{ll}
		
\textbf{appx}	&	approximately\\
\textbf{app}	&	(Mobile or web) application\\
\textbf{DevOps}	&	Development and Operations\\
\textbf{excl.}	& 	excluding\\
\textbf{Inc.}	&	incorporation\\
\end{tabular}
\end{flushleft}


%###################################################################################
%###################### Appendix B          ########################################
%###################################################################################
\newpage
%change this
\addcontentsline{toc}{section}{Glossary}
\section*{Glossary}
%%remove this
%\input{__help/latex_hinweise}

\begin{flushleft}
\begin{tabular}{ll}

\textbf{Access Key}			&	 	\ref{accesskeys}\\


\textbf{Alexa}				&		\ref{alexa:def}\\
\textbf{Alexa Skill}		&		\ref{alexa:def}\\
\textbf{Alexa Skills Kit}	&		\ref{ask:def}\\
\textbf{Amazon Developer Console}&	\ref{ask:devconsole}\\

\textbf{Application ID}		&	required by Lambda. Skill ID in our context\\

\textbf{Intent}				&		\ref{intents}\\
\textbf{Interaction Model}	&		\ref{interactionMdl:def}, \ref{interactionModel}\\



\textbf{Lambda, AWS Lambda Function}	&		\ref{aws:modules}\\
%\textbf{Lambda Function}	&		erklärung\\
\textbf{Lex, Amazon Lex}	&		\ref{aws:modules}\\

\textbf{Node.js}			&		\ref{nodejs:def}\\

\textbf{Polly, Amazon Polly}&		\ref{aws:modules}\\
\textbf{Public Service}			&	\ref{pubsvc}\\
\textbf{Slot}				&		\ref{slots}\\
\textbf{Skill ID}			&	amzn1.ask.skill.d7732837-fab2-42ff-a152-4eb0fc4ee646\\
\textbf{Transcribe, Amazon Transcribe}	&		\ref{aws:modules}\\

\textbf{Utterance}			&		\ref{utterances}\\

%\textbf{ElasticSearch}		&		erklärung\\


\textbf{Web Service}			&		e.g. AWS\\




\textbf{Bot}				&	Unless otherwise mentioned, a chatbot\\
\textbf{Hitlist}			&	erklärung\\
\textbf{Voice-first}		& \shortstack[l]{an always-on, intelligent piece of hardware, where the \\ primary interface is voice, both input and output.}\\

%\textbf{abk}				&	erklärung\\
%\textbf{abk}				&	erklärung\\
%\textbf{abk}				&	erklärung\\
%\textbf{abk}				&	erklärung\\
%\textbf{abk}				&	erklärung\\


\end{tabular}
\end{flushleft}
