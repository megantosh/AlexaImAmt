\chapter{Conclusion and Future Work}
\label{conclusion}
%#############################################################
%###################### Summary   ############################
%#############################################################
\section{Overview} %was summary
%DELETEME: put a plain summary of your work here. Summaries should be made of each Chapter beginning with Chapter~2 and ending with you evaluation. Just write down what you did and describe the corresponding results without reflecting on them.



\begin{wrapfigure}{r}{3cm}
	\caption[Clippy as an Early Chatbot] {Microsoft Office Clippy as an early attempt for an interactive Chatbot}
	\label{clippy}
%	\centering
	\includegraphics[width=3cm]{clippy} 
\end{wrapfigure}


Chatbots and voice assistants are redefining our everyday use of technology and have become an adopted trend. Security and open frameworks are still one of the concerns of this technology. % newly %becoming 
As the trend with IoT devices is consumer-driven, Alexa, as an IoT-powering device is a leading platform and advancing at a very fast pace compared to Google and Apple. %making leaps
With the ML drive in Voice Assistants in general, the ability to simulate a human customer experience is unveiling new sides of modernity. % are % becoming  thanks to  becoming more 
Special use cases of bots are also becoming widespread in hotel use. Relay by Savioke is one of the ultra-modern examples of how a system using a voice assistant can revolutionise the service industry. %replace ser



Chatbots and voice assistants are also raising debates about sexism in AI, their ability to supersede human thinking, our acclimatisation to their presence. They symbolise chances and threats. It is unlikely to develop indifference towards the VUI technology or the entrenched sentiment developed towards it. All of which show signs of a disruptive industry that brings a major change forward. And while their use to replace humans in different jobs is becoming reality, it can be regarded as a change that brings other changes with it since it also opens up unprecedented job opportunities in technical and non-technical fields. %is not surprising 
With respect to the new user experience, the ideas behind making technology have a human-friendly and familiar interface existed since the developments in GUI. A major example is the Microsoft Office 97 Clippy (Figure \ref{clippy}).  %chatbots were presen

With the ability to generate synthesised speech sounds and play custom audio, the experience becomes much more immersive.
The enriched experience applies mostly on the US-English Alexa, given the generally higher number of users and developers in the language. Localisation is a top priority in future.




The ability to develop for Alexa is a win-win scenario for developers interested in expanding an existing system. It promises high adoption of the Skill and a %ability to 
bonus of using many conversational features of Alexa as well as `small-talk', which gains fascination in commercial use.
Alexa is taking the `if this, then that'(IFTTT) concept  to voice commands at a large scale, which in turn makes expectations from technology rise beyond the hope of a user to find a solution on their own. Hence, Alexa is perceived as a helping AI for the customer and an enabler in the service industry.





\subsection*{Difference Between Lex and Alexa Skills}
\label{lexAlexa}  

%\todo{cite? indent? copied text}
Amazon Lex is a service for building conversational interfaces using voice and text. Powered by the same conversational engine as Alexa, Amazon Lex provides high quality speech recognition and language understanding capabilities, enabling addition of sophisticated, natural language \'chatbots\' to new and existing applications. Amazon Lex reduces multi-platform development effort, allowing you to easily publish your speech or text chatbots to mobile devices and multiple chat services, like Facebook Messenger, Slack, Kik, or Twilio SMS. Native interoperability with AWS Lambda, AWS MobileHub and Amazon CloudWatch and easy integration with many other services on the AWS platform including Amazon Cognito, and Amazon DynamoDB makes bot development effortless~\footnote{
	More information: \url{https://stackoverflow.com/questions/42982159/differences-between-using-lex-and-alexa\#URL}\\
	\url{https://aws.amazon.com/about-aws/whats-new/2017/09/export-your-amazon-lex-chatbot-to-the-alexa-skills-kit/}\\
	\url{https://aws.amazon.com/lex/faqs/}
}.





\section{Conclusion}


With our Skill we present an uncommon solution for an e-Government system, which promises high usage due to the many customer segments it covers.
The ASK offers a wide range of programming concepts, tools, blueprints and documentation that is constantly evolving to make development more focused  on programming logic.

Exploring the VUI paradigm through Alexa gives substantial knowledge in the field. The process is well-defined throughout the lifecycle of the software, from building to deployment to testing and publishing, in addition to benchmarking and obtaining real-time measures


.
The separation of concerns between front-end and skill logic allows teams to work in DevOps.


Attention is given to front-end and back-end elements equally, e.g. with nested handlers and entity resolution.
When designing our Skill, there are various design guides that give insight on how to retain customers. Primarily, making the content suitable for voice is the main concern when coming from a GUI approach. % makes a good 
For that a design should always start with a flow chart to highlight key breakpoints in the conversation. While it is of high value to be aware of what information Alexa can store during a session, it is also vital to estimate what information the user can keep during a conversation and not only the voice assistant.
Designing a dialogue for a Skill is therefore best done with a partner to take turns on the conversation and diversify the thinking process. %Of the various tools 

With the remarkable growth of VUI, we witness an unprecedented growth rate in adoption of the technology quicker than any modern advancement from punchcards to the command-line interface, graphical or haptic user interface.
And so, what distinguishes the quality of a %good 
system is %from a better one is 
how fast it learns.
Since languages do not work with a unified grammar and word types (adjectives, adverbs) are not simply replaceable by one another.
In that respect, using term weighting, neural networks and thought vectors have led to advancements in nuance detection especially in homophones and homographs and are well integrated in Alexa.

Worthy of mentioning is that with this advancement comes a responsibility of handling the huge amounts of data. Sound, being the most complex data set is in so far different to usual data, as it contains biometric information, which brings dangers in an unsecured cloud environment.

The Berlin Service skill developed thus far already shows promise in the use of VOI for government service delivery. Further steps with it are likely to provide a service that is truly useful for the citizens of Berlin.












\section{Future Work}

This work explores potentials for future work in various tracks. On one hand, with the knowledge collected about Alexa and the way it works, there are many possibilities to expand the developed Skill. With Amazon's introductions to the Echo Button devices, the voice-first experience becomes more expandable. %With Alexa's current support, Buttons can be used to extend sessions, which can be useful in g
Building on the concepts learned, there is a possibility to develop tools that make Skill development easier and more accessible to users with less programming background. Expanding the API base is an addition to the community that helps technology leap forward.

Finally, as VUI is concept in early adoption, there is a need to explore the security of its use. Lei et al discuss the lack of security in Alexa\cite{lei} amid many controversies on the platform. In many IoT devices powered by Alexa and Google Home, the current trend is to have an AI centralised in far away servers with data sent back and forth between continents only to turn off a light bulb. Making autonomous IoT can be a pivotal change by decentralising information collection. We learn that infringing privacy is a harm to sovereignty and still stands in the way of making the VUI technology widespread. Potential research field is how to seamlessly have a on-device VUI which is (at least not completely) cloud dependent. With commercial solutions already emerging, there is an urgent need for collaboration between the industry and academia in the field for the rising VUI community to flourish organically and in a manner not exploiting the user. Believing that data privacy is crucial to a healthy society, companies like Amazon should be also forced to become more transparent. With the upcoming GDPR, the development of VUI becomes reliant on societal consent. Educating the public about their privacy is a start.

%snips

%the entity collecting massive infor
%academic research continues to explore it in different aspects, from a psychol







%\url{http://www-personal.umich.edu/~cellis/heteronym.html}
%homophones, homographs
%moved from ch2 Underlying Grammar: since languages do not work with a unified grammar and word types (adjectives, adverbs) are not simply replaceable by one another. If a person say I a


%from ch 2:
%And so, naturally, what distinguishes a good system from a better one is how fast it learns.
%
%\todo{That's why it's good that there is more than one giant taking care of this}



%nested handlers

%\todo{
%\textbf{these are repeated form intro etc so keep it short}
%den menschlichen Aspekt suggerieren\\
%menschliches Verhalten immitieren\\
%smalltalk fähigkeiten\\
%these are centralized at alexa somewhere\\
%ablitiy to react to everything\\

%Handyversicherung;\\
%from wiInf perspective, the bot is aiming to sell more versicherung, \\
%the bot tries to determine if there is a ekhtelaf fel egabat (acting as a judge)\\
%MKTG - Aufwand - how did the phone fall off,\\
%- use of ML
%
%unfortunately forums vs. FAQs did not work. if i want assistance, i want the customer to tell me the model number - and forums have mostly Schrott!
%
%
%alexa is an IFTTT for voice commands?
%}



%	-- chatbots as enablers in customer service industry\\
%	-- conclusion: Although not impossible, it is a bit too 


%\todo{
%at alexa somewhere \textbf{i.e. talk about SKILLS}\\
%- ability to retain sessions (explain requests/responses - GET/POST)\\
%- fullfilling intents\\
%- nested handlers\\
%\\
%skill service: code - business logic - handles json requests
%skil interface: configuration (developer portal)
%\\
%- difference to Lex \& Polly : \href{https://stackoverflow.com/questions/42982159/differences-between-using-lex-and-alexa}{diff alexa lex} \\
%
%- Major prob: lex is not in german\\
%- \href{https://www.youtube.com/watch?v=QxgdPI1B7rg}{Alexa Documentation}
%}





%%====%====%====%====%====%====%====%====%====%====%====%========


%mention the D115 how u met them

%IMPORTANT
%table with fara3in (Nodes):
%\todo{non-fluid text, as in not human readable. we see what it means later with the SSML, add string in line}
%




%kurze sätze to Alexa 3ashan mateghlatsh


%#############################################################
%###################### Conclusion ###########################
%#############################################################
%\section{Conclusion}
%DELETEME: do not summarize here. Reflect on the results that you have achieved. What might be the reasons and meanings of these? Did you make improvements in comparison to the state of the art? What are the good points about your results and work? What are the drawbacks? 

%there has always been an attempt to make technology have a familiar interface;
%Office clippy 
%MSN dog. 
%(pics!)

%we now have hotel bots


























%
%
%%choice of voice; Female vs male voice- persönlich Mann mütterlich Ikea 
%%
%%
%our influence continues 
%some depict current technology with a sceptic notion
%as ~ shows in his show 'black mirror'
%the trend of being scared of AI exists and should not be underestimated as it is not like with the introduction of phones. There are many situations where AI can help us. Autism, blindness
%
%
%People get more used to it and do get educated but life becomes effectively more difficult but more effectively (we will do more)
%
%
%
%
%\todo{
%
%
%-matensash el evaluation beta3et el chatbot\\
%
%-	kritikk that amazon is v haphazard. zB memo in workshop\\
%
%\href{http://voicelabs.co/2017/02/08/interview-with-sean-fisher-retention-dos-and-donts/}{Retention Dos and Don'ts}
%
%}
%
%
%\todo{
%beyond the features, we can take something from wienbot:
%they are active about marketing their bot
%
%mention what aspects of the model we can mock. e.g. their \@wienbot twitter feed re new launches.
%\url{https://futurezone.at/digital-life/wienbot-die-stadt-wien-hat-einen-chatbot/246.709.043}
%
%}
%
%
%
%
%
%%#############################################################
%%###################### Future Work ##########################
%%#############################################################
%\section{Future Work}
%%DELETEME: Regarding your results - which problems did you not solve? Which questions are still open? Which new questions arised? How should someone / would you continue working in your thesis field basing on your results?
%\textcolor{magenta}{
%- use machine learning to rank higher demands for more popular services.\\ 
%- matkhoshesh fel 7etta di awi - for now hitlist already given.\\
%- future of bots. deren Einsatz. roles (As judges, catereres in hotels (that hotel botler) \\
%-\\
%}
%
%\todo{
%We are going to build neurons so they can bridge together (Alexa Podcast 18)
%Otherwise they will forget 
%Branches stop growing when they reach another branch
%Intelligenceaggreation - amplification 
%We are voice-first approach (as said earlier: Inner voice)
%We do not know much about human brain /intelligence-
%-the point is not in taking over but making Internet more powerful (Podcast)
%information will appear as you need it: 
%Voice first not voice only - less time to look at screen. 
%}
%
%
%
%
%
%%moved from a footnote
%This derives into a possible scenario where Alexa's AI becomes smarter then maybe in a few years the concept of Skills won't exist as a stand-alone any more and would be integrated into Alexa's own brain without the user knowing and it would just be a hit or miss kind of thing.
%
%
%
%Meyers bricks/ need to be on time/ahead of time
%Machines replace us where we are bad or don’t want to do sth
%Custom tailored system
%
%
%Recognize tone and act accordingly 
%
%
%
%
%
%
%
%Educate yourself about AWS role in government, e.g. 
%AWS public sector summit, Washington D. C., June 20-21, 2018
%\url{https://aws.amazon.com/summits/public-sector-summit-washington-dc-2018/?sc_channel=em&sc_campaign=EMEA_WWPS_NL_city-on-a-cloud-innovation-challenge_2018051&sc_medium=em_86380&sc_content=REG_nl_wwps&sc_geo=emea&sc_country=de&sc_outcome=reg&trk=em_86380&trkCampaign=aws-ps-summit-dc-2018-de-cityonacloud_em&trk=aws-ps-summit-dc-2018-de-cityonacloud_em&mkt_tok=eyJpIjoiTVdVNU9UZzVORGsyWVdFMCIsInQiOiJ5WVdNa0t1MytnVnEyUDgwXC9WZjdrU3ltdnhYU0t6dzYrTWxZcFlESVpCUnB5T1B0ekV1VFFRNmpyV0Nsblo5Vm5FaXhGbmV6UzhRNHZtdEltWnBjNmthRXpLZTY4RVNOOGJCMjNSUjg0XC9KbGZlOVRVNmM1d1F2UlhHTFBvRDhCV2d2UzY3bEdYNFJNYVRyQ2NmYTg5QT09In0%3D}
%	
%	
%	
%	
%	
%	\todo{check bibstyle - check all relevant docs in Lit review folder are included\\
%	angabe of events i went to: summit and dev day (workshop) http://alexadevday.com/BER/DevPortal\\}