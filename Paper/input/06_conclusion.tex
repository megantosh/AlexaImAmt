\chapter{Conclusion and Future Work}
\label{conclusion}
%#############################################################
%###################### Summary   ############################
%#############################################################
\section{Overview} %was summary
%DELETEME: put a plain summary of your work here. Summaries should be made of each Chapter beginning with Chapter~2 and ending with you evaluation. Just write down what you did and describe the corresponding results without reflecting on them.



Chatbots and voice assistants are raising debates about sexism in AI, their ability to supersede human thinking, our acclimatisation to their presence. They symbolise chances and threats and are unlikely to be indifferent to. All of which show signs of a disruptive industry that brings a major change forward. And while their use to replace humans in different jobs is becoming reality, it can be regarded as a change that brings other changes with it since it also opens up unprecedented job opportunities in technical and non-technical fields. %is not surprising 



%	-- chatbots as enablers in customer service industry\\
%	-- conclusion: Although not impossible, it is a bit too 




\todo{

at alexa somewhere \textbf{i.e. talk about SKILLS}\\
- ability to retain sessions (explain requests/responses - GET/POST)\\
- fullfilling intents\\
- nested handlers\\
\\
skill service: code - business logic - handles json requests
skil interface: configuration (developer portal)
\\
- difference to Lex \& Polly : \href{https://stackoverflow.com/questions/42982159/differences-between-using-lex-and-alexa}{diff alexa lex} \\

- Major prob: lex is not in german\\
- \href{https://www.youtube.com/watch?v=QxgdPI1B7rg}{Alexa Documentation}
}


suitable

mention the D115 how u met them

IMPORTANT
table with fara3in (Nodes):
\todo{non-fluid text, as in not human readable. we see what it means later with the SSML, add string in line}





kurze sätze to Alexa 3ashan mateghlatsh

\url{http://www-personal.umich.edu/~cellis/heteronym.html}
	homophones, homographs
	moved from ch2 Underlying Grammar: since languages do not work with a unified grammar and word types (adjectives, adverbs) are not simply replaceable by one another. If a person say I a
	
	
	from ch 2:
	And so, naturally, what distinguishes a good system from a better one is how fast it learns.
	
	\todo{That's why it's good that there is more than one giant taking care of this}
	
	

%#############################################################
%###################### Conclusion ###########################
%#############################################################
\section{Conclusion}
%DELETEME: do not summarize here. Reflect on the results that you have achieved. What might be the reasons and meanings of these? Did you make improvements in comparison to the state of the art? What are the good points about your results and work? What are the drawbacks? 



\todo{


-matensash el evaluation beta3et el chatbot\\

-	kritikk that amazon is v haphazard. zB memo in workshop\\

\href{http://voicelabs.co/2017/02/08/interview-with-sean-fisher-retention-dos-and-donts/}{Retention Dos and Don'ts}

}


\todo{
beyond the features, we can take something from wienbot:
they are active about marketing their bot

mention what aspects of the model we can mock. e.g. their \@wienbot twitter feed re new launches.
\url{https://futurezone.at/digital-life/wienbot-die-stadt-wien-hat-einen-chatbot/246.709.043}

}





%#############################################################
%###################### Future Work ##########################
%#############################################################
\section{Future Work}
%DELETEME: Regarding your results - which problems did you not solve? Which questions are still open? Which new questions arised? How should someone / would you continue working in your thesis field basing on your results?
\textcolor{magenta}{
- use machine learning to rank higher demands for more popular services.\\ 
- matkhoshesh fel 7etta di awi - for now hitlist already given.\\
- future of bots. deren Einsatz. roles (As judges, catereres in hotels (that hotel botler) \\
-\\
}

\todo{
We are going to build neurons so they can bridge together (Alexa Podcast 18)
Otherwise they will forget 
Branches stop growing when they reach another branch
Intelligenceaggreation - amplification 
We are voice-first approach (as said earlier: Inner voice)
We do not know much about human brain /intelligence-
-the point is not in taking over but making Internet more powerful (Podcast)
information will appear as you need it: 
Voice first not voice only - less time to look at screen. 
}





%moved from a footnote
This derives into a possible scenario where Alexa's AI becomes smarter then maybe in a few years the concept of Skills won't exist as a stand-alone any more and would be integrated into Alexa's own brain without the user knowing and it would just be a hit or miss kind of thing.