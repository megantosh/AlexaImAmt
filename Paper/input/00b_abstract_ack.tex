%#############################################################
%###################### Statement ############################
%#############################################################
%\chapter*{Erkl{\"a}rung der Urheberschaft}
%%this one needs to be signed for submission
%Ich erkläre hiermit an Eides statt, dass ich die vorliegende Arbeit ohne Hilfe Dritter und ohne Benutzung anderer als der angegebenen Hilfsmittel angefertigt habe; die aus fremden Quellen direkt oder indirekt übernommenen Gedanken sind als solche kenntlich gemacht. Die Arbeit wurde bisher in gleicher oder ähnlicher Form in keiner anderen Prüfungsbehörde vorgelegt und auch noch nicht veröffentlicht.
%
%
%\vspace{4cm}
%
%Ort, Datum \hfill Unterschrift\\
%Berlin, den \today


%#############################################################
%###################### Abstract  ############################
%#############################################################
\newpage
\chapter*{Abstract}
%preliminary
%DELETEME: An abstract is a teaser for your work. You try to convince a reader that it is worth reading your work. Normally, it makes to structure you abstract in this way: 
%\begin{itemize}
%\item one paragraph on the motivation to your topic
%\item one paragraph on what approach you have chosen
%\item and one paragraph on your results which may be presented in comparison to other approaches that try to solve the same or a similar problem.
%\end{itemize}
%Abstract should not exceed one page (aubrey's opinion)

Though not a recent phenomenon, chatbots and voice assistants are increasingly gaining unprecedented attention as a successor for mobile and web apps. While still emerging with no defined standards or set protocols, with a hype on the rise, tensions between industry giants with products like Amazon's Alexa, Apple's Siri or the Google Assistant %or IBM's Watson 
unveil new examples for providing an enriched user experience for consumers and businesses. The surrounding ecosystem also plays a major role in spreading the platforms available while exploring new horizons with alternative approaches and business models. Today voice assistance are already present around indoor spaces, in the car or on the go. They live in smartphones, on fancy TV sets or even as stand-alone devices. Yet, they are still a new terrain to discover and great potential to unleash in numerous contexts.\\ 

One such scenarios involves the public sector. In this work, we explore the Alexa Skills Kit by Amazon in combination with respective services to develop a voice assistant for the local city council extending the current chatbot's functionality available on \href{ https://service.berlin.de/virtueller-assistent/virtueller-assistent-606279.php}{http://service.berlin.de}. We touch on the technical challenges and possibilities in implementing a Software as a Service (SaaS) system for e-Government inquiries and analyse its usability as well as its effectiveness in replacing traditional information lookup. We then examine the goals we define for our use case to our achievement with the APIs and SDKs available at the time. With respect to those, we also report on the opportunities, challenges and limitations faced in the development process.\\

Finally, we study the current state of voice assistants and service bots on the market and the future of this trend from a technical and a social point of view.

\sn{conclusion: Das Ergebnis der Arbeit sollte man zum Schluss noch im Abstract ergänzen.}
%\note{Im Abstract sollte man aim/will/going to meiden. Besser: Einfach Präsens oder sogar Perfekt}
%\note{Ich finde es besser, im Präsens zu schreiben: Also besser: We analyze oder We study (anstatt von going to)} 


%#############################################################
%###################### German Abstract ######################
%#############################################################

\newpage
\chapter*{Zusammenfassung}
%\note{contextual translation to German.}

Auch wenn Chatbots und Sprachassistenten keine Randerscheinung mehr sind, erlangen sie in letzter Zeit eine steigende Bedeutung als Nachfolder von mobilen Anwendungen und Web Applikationen. Trotz des Mangels an definierten Standards oder feste Protokolle, mit einer zunehmenden Aufmerksamkeit auf dem Markt herrscht große Konkurrenz. Ob es durch Amazons Alexa, Apples Siri oder den Google Sprachassistenten hervorgeht, zeigen sich zahlreiche Beispiele, wie die Nutzererfahrung auf privaten oder Geschäftsebene weitgehend bereichert werden kann. Mit Lock-in Effekten spielen Drittanbieter eine große Rolle bei der Verbreitung der einen oder anderen Plattform und sorgen für neue Interaktionsmodelle. Der steigende Nutzen von Sprachassistenten zuhause, im Büro und unterwegs erlaubt viele neue Erkenntnisse und Erfahrungen, die dem Marktumfeld zu verdanken sind. \\


Auch im öffentlichen Dienst wird dieser Nutzen vertreten und wertvoll eingeschätzt. In der folgenden Recherche untersuchen wir Alexa von Amazon im Zusammenhang mit anderen verfügbaren Plattformen und erweitern die Fähigkeiten des Sprachassistenten durch den Webauftritt des berliner Stadtportals (BerlinOnline) um weitere Dienste sowohl für die vielfältigen und mehrsprachigen Einwohner der Hauptstadt als auch deren Senatsverwaltung. Als Grundlage wird auf die Funktionalität des Chatbots auf dem Stadtportal \href{ https://service.berlin.de/virtueller-assistent/virtueller-assistent-606279.php}{http://service.berlin.de} zurückgegriffen. Anhand dieses Anwendungsfalls werden die prinzipiellen Möglichkeiten und Herausforderungen bei der Implementierung eines Sprachassistenten für die Stadtverwaltung aus technischer und organisatorischer Sicht analysiert. Zusätzlich wird die Usability des Systems als Ergänzung für einen Abfragedienst in Erwägung gezogen. Anschließend werden die erreichten Ziele mit den gesetzten Anforderungen in Perspektive gestellt. Darüber hinaus wird über Systemgrenzen sowie individuellen und systematischen Fehlerquellen berichtet, die bei der Entwicklung behandelt worden sind.\\


Schließleich nehmen wir den aktuellen Trend von Sprachassistenten und Chatbot-Dienste unter der Lupe und diskutieren eventuelle Zukunftentwicklung mit dem gegeben Stand jenseits des e-Governance Bereichs.
\sn{conclusion: Das Ergebnis der Arbeit sollte man zum Schluss noch im Abstract ergänzen.}


% outline, proposal, design specs

%#############################################################
%###################### Acknowledgements #####################
%#############################################################

\newpage
\chapter*{Acknowledgements}
%DELETEME: Thank you for the music, the songs I am singing

\inote{possibility to make inline notes}
\citemissing{when a citation is missing}
\todo{todos}
\tocite{to cite} \\
\\
\\
\sn{Bitte Betreuernotizen auch im Fließtext  mit \textbackslash sn\{text\} }
