%#############################################################
%###################### Statement ############################
%#############################################################
%\chapter*{Erkl{\"a}rung der Urheberschaft}
%%this one needs to be signed for submission
%Ich erkläre hiermit an Eides statt, dass ich die vorliegende Arbeit ohne Hilfe Dritter und ohne Benutzung anderer als der angegebenen Hilfsmittel angefertigt habe; die aus fremden Quellen direkt oder indirekt übernommenen Gedanken sind als solche kenntlich gemacht. Die Arbeit wurde bisher in gleicher oder ähnlicher Form in keiner anderen Prüfungsbehörde vorgelegt und auch noch nicht veröffentlicht.
%
%
%\vspace{4cm}
%
%Ort, Datum \hfill Unterschrift\\
%Berlin, den \today


%#############################################################
%###################### Abstract  ############################
%#############################################################
\newpage
\chapter*{Abstract}
%preliminary
%DELETEME: An abstract is a teaser for your work. You try to convince a reader that it is worth reading your work. Normally, it makes to structure you abstract in this way: 
%\begin{itemize}
%\item one paragraph on the motivation to your topic
%\item one paragraph on what approach you have chosen
%\item and one paragraph on your results which may be presented in comparison to other approaches that try to solve the same or a similar problem.
%\end{itemize}
%Abstract should not exceed one page (aubrey's opinion)

Though not a recent phenomenon, chatbots and voice assistants are increasingly gaining unprecedented attention as a successor for mobile and web apps. While still emerging with no defined standards or set protocols, with a hype on the rise, tensions between industry giants with products like Amazon's Alexa, Apple's Siri, the Google Assistant %or IBM's Watson 
unveil new examples in favour of providing an enriched user experience on consumer and business level. The surrounding ecosystem also plays a major role in widening the platforms available while exploring new horizons with alternative approaches and business models. Today voice assistance are already present around indoor spaces, in the car or on the go but are still a new terrain to discover and great potential to unleash. 

One such use cases involves the public sector. In this work, we are going to explore Amazon's Alexa and respective platforms to develop a voice assistant for the local city council extending the current chatbot's functionality available on \href{ https://service.berlin.de/virtueller-assistent/virtueller-assistent-606279.php}{http://service.berlin.de}. We will touch on the technical challenges and possibilities in implementing a system for eGovernment inquiries and touch on its usability as well as effectiveness in replacing a traditional lookup service. We will then examine the goals we define for our use case to what we were able to achieve with the available APIs and SDKs. With respect to those, we will also report on the limitations developers could face in the process.

Finally, we aim at analysing the current state of voice assistants and service bots in the market and the future of this trend from a technical and a social point of view.


%#############################################################
%###################### German Abstract ######################
%#############################################################

\newpage
\chapter*{Zusammenfassung}
\note{translate to German to English or vice-versa.}
\inote{possibility to make inline notes}
\citemissing{when a citation is missing}
\todo{todos}
\tocite{to cite} \\
\\
\\
\sn{Bitte Betreuernotizen auch im Fließtext  mit \lstinline|\sn{text}| }

%#############################################################
%###################### Acknowledgements #####################
%#############################################################

%\newpage
%\chapter*{Acknowledgements}
%DELETEME: Thank you for the music, the songs I am singing