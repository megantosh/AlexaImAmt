%#############################################################
%###################### Statement ############################
%#############################################################
%\chapter*{Erkl{\"a}rung der Urheberschaft}
%%this one needs to be signed for submission
%Ich erkl�re hiermit an Eides statt, dass ich die vorliegende Arbeit ohne Hilfe Dritter und ohne Benutzung anderer als der angegebenen Hilfsmittel angefertigt habe; die aus fremden Quellen direkt oder indirekt �bernommenen Gedanken sind als solche kenntlich gemacht. Die Arbeit wurde bisher in gleicher oder �hnlicher Form in keiner anderen Pr�fungsbeh�rde vorgelegt und auch noch nicht ver�ffentlicht.
%
%
%\vspace{4cm}
%
%Ort, Datum \hfill Unterschrift\\
%Berlin, den \today


%#############################################################
%###################### Abstract  ############################
%#############################################################
\newpage
\chapter*{Abstract}
%preliminary
%DELETEME: An abstract is a teaser for your work. You try to convince a reader that it is worth reading your work. Normally, it makes to structure you abstract in this way: 
%\begin{itemize}
%\item one paragraph on the motivation to your topic
%\item one paragraph on what approach you have chosen
%\item and one paragraph on your results which may be presented in comparison to other approaches that try to solve the same or a similar problem.
%\end{itemize}
%Abstract should not exceed one page (aubrey's opinion)

Though not a recent phenomenon, chatbots and their use cases are on the rise in a still emerging market with no defined standards or set of protocols. With the increasing hype comes tensions within and between tech giants and the rest of the industry about providing a complete user experience, recruiting more developers, widening a platform and exploring new horizons that claim to make the consumer's live easier while finding a business opportunity at the same time. And while smaller enterprises use the same circumstance as an incentive and an attempt to grow, chatbots come today in several shapes and sizes but are still a new terrain to discover with their usability, their social impact and their business models.

In this work, we are going to explore different approaches to making chatbots, their presence around the web as well as their trending markets. As a focus, we will present the possibilities available on multiple platforms including Amazon Alexa and Facebook Messenger with a voice and text service bot for public inquiries based on Berlin's official city portal \url{https://www.berlin.de}. We will then examine its usability and compare the goals we define for our use case to what we were able to achive with the available APIs and SDKs. With respect to these, we will also touch on the challenges and limitations developers face in the process.

Finally, we aim at analyzing the current state for virtual service assistants and the future of this trend from a technical and a social point of view


%#############################################################
%###################### German Abstract ######################
%#############################################################

%\newpage
%\chapter*{Zusammenfassung}
%DELETEME: translate to German to Englisch or vice-versa.

%#############################################################
%###################### Acknowledgements #####################
%#############################################################

%\newpage
%\chapter*{Acknowledgements}
%DELETEME: Thank you for the music, the songs I am singing