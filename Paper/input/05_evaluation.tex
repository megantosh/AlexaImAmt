\chapter{Evaluation}
\label{evaluation}
%DELETEME: The evaluation chapter is one of the most important chapters of your work. Here, you will prove usability/efficiency of your approach by presenting and interpreting your results. You should discuss your results and interprete them, if possible. Drawing conclusions on the results will be one important point that your estimators will refer to when grading your work.







After a completed implementation with all necessary intents and their handlers included, we measure the Skill's performance against our expectations at the early stages of the project, then compare these to WienBot and GeorgiaGov, which we take as qualitative benchmarks.



\section{Compliance}

With D115's promises to its customers, we revisit each of the goals and see how we implemented these:
%\todo{
%how it complies to D115 promises, from data protection to alles andere}


\begin{enumerate}
	\item \textbf{24/7 availability:} Our Skill server and Lambda are not tied to specific opening hours as opposed to D115. 
	
	\item\textbf{ Avoiding office visits:} While we initially depend on the information on the Berlin City Portal, we expect that customers will have unhandled answers at this early stage. For that we implemented the \mintinline{text}{unhandled} intent, which tells us what strange formulations and converations happened anonymously.
	
	\item \textbf{Freedom and accuracy of information: } The voice assistant as a solution gives explanations as provided by the website. Hence, information accuracy is guaranteed. This information relates to the details around the public service, such as fees, processing time. While implementing answers to the legal base is possible, we did not see in the conducted survey a necessity for it and could add it as a feature in a later version, which speaks for the flexibility of the voice assistant solution. % did not implement a  %and % to make it conform to the Right to Information (Article 19 of the Universal Declaration of Human Rights) in accordance with the German Freedom of Information Act - Gesetz zur Regelung des Zugangs zu Informationen des Bundes~\footnote{\url{ https://www.gesetze-im-internet.de/englisch_ifg/index.html}}.
	
	\item  \textbf{Improvements: } With the ability to update the software only from back-end side, users do not need to carry out updates themselves. Hence, potentially incomplete information can be eliminated quickly without the user to take action. The only bottleneck is Amazon's approval, which takes one to two days as stated on their website. Monitoring of their Twitter feed, this seems to be the case. % to the voice service offering through learning from new specific cases that can be helpful to solve others in the future.
	
	\item \textbf{Helpfulness and Friendliness: } With Alexa already having a positive reputation about being helpful, our Skill extends this character in several ways. The German `Sie' Form adds a touch of formality, which is expected within the scope of a public service. Implementing a \mintinline{text}{HelpIntent} and contextual help extends the user's perception of this helpfulness. There is also a possibility with Alexa to be set to speaking short sentences if the user enables this configuration. From programming side we can control texts spoken out in the short and long form, giving more detail in the latter or keeping the minimum necessary information (e.g. in keywords format) in the former. %gives the user this 
	With the ability to keep questions short by asking directly about a certain aspect (e.g. required documents) of the service, this gives user a more well-rounded and understanding experience.
	
%	\item A service that is friendly and respectful of its target audience
	\item \textbf{Trigger Inquisitiveness: } Our Skill ends almost every answer with another question triggering the user whether they want to find out more information about another sub-topic.
%	\item An open option to equip customers with more information by triggering them to ask for, in case it could be helpful without them knowing at the time of inquiring. % for the information
	
	\item \textbf{Anonymity:} With no personal session details stored, we do not have any influence over what Alexa can say to one certain device and we do not collect information about the user. The only information collected is anonymised and time-shifted logs about the use of the Skill only, such as what intents were used. 
%	\item A service that gives facts anonymously to the customer without collecting data that can potentially result in any information bias at the time of service or in the future
	\item \textbf{Flexibility: } With the option to ask subcategorical quesions, and move between question topics freely with the \mintinline{text}{StopIntent} or after the question has been answered, the user can interrupt Alexa as they would not on the D115 hotline. Alexa's availability also delivers longer uptime for the user, as opposed to a hasty behaviour of on the phone. %when a call can last for a short
%	\item A flexible service that can respond to the questions answered through analysing the relevant information and omitting unnecessary details.
	
	
	
\end{enumerate}





\section{Comparison with Existing Systems}

In order to obtain an appropriate position on the market, we use the reference systems as qualitative benchmarks. The following demonstrates a couple of observations made.

\subsection*{WienBot}

WienBot is a chatbot that uses speech recognition technologies, hence  not a voice-first solution with a screen always present as opposed to Alexa. Therefore the comparison remains mostly on language.
On one hand, at the time of this research, WienBot was not able to understand synonyms. Using the word `Perso' for `Personalausweis' in German (Germany and Austria) should be understandable. This could either mean that the system was not configured to make sense of synonyms or that this particular word was not registered despite its common use as a dialect.
As we expect users to use natural language, we consider not only formal language (Hochdeutsch).


With respect to helpfulness, Wienbot searches the web for input it does not have in its own database and does not stop when it cannot resolve the question.


On the other hand, while it does not compare with the use of our Skill, WienBot understands certain Emojis and reacts funnily to them. It also speaks a shorter text than the multiple callouts displayed on the screen. This gives an incentive to understand that the user is not always interested in extensive information. For instance, users do not go through the first few search result pages on a screen, but directly try the first result and check its relevance themselves. If not, they check the second. This is the behaviour WienBot tries to simulate.

Alexa in turn, does not display information on a card that is extra and necessary at the same time. We diverge display from speech so far only in the help menu, where the user still has a way to navigate through the help menu by voice and does not risk a loss of information.

As for anonymity, the makers of the chatbot denote in their fact sheet that ``the data protection component in the WienBot app should also be highlighted as it complies with the high standards of the city of Vienna''. No further information was given.

%\todo{
%%	WienBot - der Chatbot der Stadt Wien.pdf\\
%
%-versteht nicht, wenn ich `perso' schreibe\\
%-verbose.. "warum möchten Sie aus der Stadt...\\
%-uses emojis\\
%-speech recognition happens on device (iOS)\\
%-use failed attempts of screenshots (new and old)
%\\
%Spricht was anderes\\
%Versteht keine emojis\\
%Nur scheisse\\
%Dann spricht er nicht\\
%
%}



\subsection*{GeorgiaGov}


It prevails from our use of the voice assistant, that sessions are relatively long for the use we tested it for (16 sec vs. 8 sec). Comparatively, our Skill makes use of an extended session only when it is necessary. The necessity was determined through A-B tests on two users during the development process. % a variable 
According to the Amazon Skill page, there is no user-specific data stored. Only dynamic content is generated, meaning that the Skill makes API calls as we described in Section \ref{georgiastuff}, which also applies to us, guaranteeing in both cases that the information is up-to-date. 

Moreover, intent chaining is taken advantage of with Alexa asking whether related question to a topic of interest should be spoken out to the user. We do not apply this feature for this development stage.
%- Talk about features like Maintaining context(sesions), \textbf{Intent chaining}\\

With respect to synonyms, canonical value detection and slot elicitation, in some intent invocations, Alexa was unable to understand when a term is common between two intents, such as `child', which we find on the website occurring in the topic `child care' and `child custody'. Our Skill is equipped with slot elicitation directly through the interaction model. The prompts are handled in the back-end in German and English separately. As slot elicitation is a concept recently introduced by the Alexa developer team, its potential is not present in many Skills yet. Figure \ref{astah:georgiaActivity} shows our understanding of the GeorgiaGov flow of conversation.
%child care
%vs child custody. Sorry i didnt get that. (check first skizze)





The organisation of the GeorgiaGov Skill differs to Berlin.de's structure as follows:

Services are grouped into categories as seen on \href{http://www.georgia.gov}{georgia.gov}.
Alexa catches from any sentence you say only the service grouping name, then gives a definition about the service. Afterwards it asks if the user wants to hear more about the related service (which corresponds to the FAQ section on the website). The FAQs are formulated not in imperative form and have a more use-case-based wording. Example: ``I am 17 and my parents are blind. How do I renew my driving licence as a minor?''  The user replies only with yes/no answers. We infer that Acquia made its own implementation of the \mintinline{text}{YesIntent} and \mintinline{text}{noIntent}. Our Skill model has a different approach. The users can proceed with this tree-based understanding of services, giving a definition then asking about subcategories. Additionally, they can also cut the conversation short by asking directly about subtopics, e.g. ``What are the required docments for \textless service\_name\textgreater. Hence, hierarchical graph design as well as frames are enforced. 

Finally, contextual help is not implemented in the GeorgiaGov Skill. Reprompts were not found useful and too repetitive for an interactive conversation, since the same text is spoken out without any nuanced utterance.
 %in GeorgiaGov Skill, 
 Conversely, every intent ends with a question whether the user wants to have a phone number for the department related to handling the service. If none is found in its database, the Skill returns a general inquiries hotline (equivalent to D115). A similar implementation can be added to our Skill.



\section{Skill-Specific Metrics}

With Alexa Skills Kit providing various option to measure the Skill performance  respective to session length, customer retention rate among other features, before launching the Skill, there is the option for beta-testing through email invitations. Any user with an Amazon account and an Alexa-enabled device can be a tester. Figure \ref{evalz1} show some of the available metrics. Noteworthy, further metrics are unlocked through the SDK once the user base gets above 10.

Remarkably, the major drawback with the used API is the latency it causes due to the length of the JSON responses, which can be up to 1.5 Megabytes in size.


\begin{figure}[p]
	
	\caption[Conversation Flow on Georgia.gov]{Drafted activity Diagram of Conversation Flow on Georgia.gov Skill with frames representation}
	\centering %with 0.992 it is just in
	\includegraphics[ height=0.988\textheight]{Georgia2}
	\label{astah:georgiaActivity}
\end{figure}

%
%\todo{
%-This is how their model works:\\
%services are grouped into categories as seen on \href{http://www.georgia.gov}{georgia.gov}\\
%-Alexa catches from any sentence you say only the service grouping name.\\
%-then gives you a definition\\
%-then asks you if you want to hear about related services (in the format of FAQ)\\
%-possible examples: ``I am 17 and my parents are blind. How do I renew my license?''\\
%-it goes through this list and you can say yes and no if you want to hear about the `related' item.\\
%-of the related items are questions like `how much does it cost', `how much time does it take'. These are no explicitly put in the questions you can ask but they become clear if you have time to make conversation. (Kritikpunkt!)
%-then when it reaches the end of the list it says that it's over and if you want to have the phone number that is related tot this service grouping
%%
%%
%%
%-then the conversastion either ends or restarts.\\
%-you can ask questions only about service groupings\\
%%
%%
%-stop and cancel intent:
%\textit{Feel free to ask another question or say exit}
%-helpIntent:
%Ask me a question
%%
%repeat does
%%
%-reprompts are not very ueful and repetitive
%}

%
%This way, if a user does not see this service is relevant to them, they will say no and skip to the next



%\clearpage

\begin{figure}[p]
	\centering
	
	\caption[Skill Metrics]{Metrics for the produced Alexa Skill}
	
	\label{evalz1}
	
	
	\begin{subfigure}[b]{0.9\linewidth}
		%		\caption[Global Monthly Active Users (Social Media vs. Messaging Apps]{Global Monthly Active Users \\ in Millions \cite{businsider}}
		\includegraphics[width=\linewidth]{metrics/0}
	\end{subfigure}
	
	
	
	
	\begin{subfigure}[b]{0.9\linewidth}
		%		\caption[Global Monthly Active Users (Social Media vs. Messaging Apps]{Global Monthly Active Users \\ in Millions \cite{businsider}}
		\includegraphics[width=\linewidth]{metrics/1}
	\end{subfigure}
	
	

	
	
	
	
	
	
\end{figure}




%
%
\begin{figure*}[t]
	\centering
%
%
\begin{subfigure}[b]{0.9\linewidth}
	%		\caption[Global Monthly Active Users (Social Media vs. Messaging Apps]{Global Monthly Active Users \\ in Millions \cite{businsider}}
	\includegraphics[width=\linewidth]{metrics/2}
\end{subfigure}




\begin{subfigure}[b]{0.9\linewidth}
	%		\caption[Global Monthly Active Users (Social Media vs. Messaging Apps]{Global Monthly Active Users \\ in Millions \cite{businsider}}
	\includegraphics[width=0.9\linewidth]{metrics/3-1c}
\end{subfigure}



%
\end{figure*}

%\begin{figure}[h!]
%
%%\begin{subfigure}[b]{\textwidth}
%%	%		\caption[Global Monthly Active Users (Social Media vs. Messaging Apps]{Global Monthly Active Users \\ in Millions \cite{businsider}}
%%	\includegraphics[width=10cm]{metrics/3-1}
%%\end{subfigure}
%%
%
%
%
%\end{figure}

%\subsection{Performance}

%\subsection{Benchmarks}



%\subsection{Dis-/Advantages}



\clearpage


\subsection*{Alexa as a VUI and a Developer Platform}

Throughout the development process, the system proved to have a high accuracy in understanding most common words. Specific terms or foreign dialects (mainly non-english names) are impossible to transcribe if they do not exist in Amazon's NLU engine. Even with the most common command, playing music \cite{experian}, many track, artist and album names are not found.

Likewise, an important criterion about reusing a Skill is starting the Skill, which happens through the invocation name. After settling for `Berlin Service' as a name, the Skill did not fail to invoke. Hence, choice of invocation name needs to be assessed during and after the development process.

%Experian of Alexa as a whole \url{https://www.experian.com/innovation/thought-leadership/amazon-echo-consumer-survey.jsp}


In General, Skills seem to expand in various domains. From a marketing perspective, there still remains the question, how they can monetise to provide an incentive to developers and business.
%
%\todo{are Skills really going to make it?
%	
%	30,000 users a day in the US\\
%	\url{https://omr.com/en/amazon-alexa-skill-marketing/}
%	10,000 BVG
%
%}
%
%\todo{ metrics from Seif's BA!}
%
%
%\todo{sometimes voice will hardly or never get what you want. try to tell alexa to play YAS. will always redirect you to something else}
%
%
%\todo{- wo hilft mir alexa, was mach ich lieber woanders?\\
%	- wie kann man die G\"ute des Systems beurteilen? \textbf{do not forget the surveyu made}   }
%
%
As such, Alexa is a helpful tool in most VUI use cases discussed in Chapter \ref{vui}. Since Amazon is expanding with the product in cooperation with car makers, their user base is expected to grow and Skills alike.


%\section{Metrics}

%\todo{
%\textbf{talk about umfrage-design:}\\
%skalen, yes no, choosing between limited answers, etc.\\
%combining logic and validation, such that you would be asked about only one thing if you answered the other respectively.
%done in english and german for purpse: cater for different people living in berlin\\
%conjoint analyse\\
%a-b testing - see what users prefer then come to conclusion that dienstleistung is the most important\\
%we evaluate how people speak to alexa with sample sentences.\\
%ask questions like "Did u know berlin.de has a behindertenberechtigt" to educate the person at the same time\\
%}

%\todo{
%what disturbs you  most about alexa?\\
%would u try it again if the question fails?\\
%-thats why the retention rate is low
%}

%As with any product, the customer expectation about accuracy is still high

%
%\todo{
%NPR\\
%time series
%a-b testing
%weighting
%use of correlations in the legal system (got, defaulting)
%search map predicts ur personality
%fear of judging, fear,…interdependence with robots
%likelihood of having different social cirlces for a lasting relationship
%\\
%- see how many use the skill after publishing
%tinyurl.com/AlexaBln
%\url{https://umfrage.hu-berlin.de/index.php/879458?lang=en}
%\url{https://umfrage.hu-berlin.de/index.php/879458?lang=de}
%}


\section{Summary}


We were able to reuse some of the artefacts from the Virtual Citizen Assistant. In order to provide a comprehensive software, content needs to be maintained by hand at the beginning as with most other systems. The drawback of having to talk to the device in only one language cannot be changed by the Skill, despite it being able to speak in more than one language with the same language setting. % The Skill can speak  
adopting the invocation name by a high user base is one of the key performance indicators to the Skill's success in commercial use.
With the help of the previously built systems Virtual Citizen Assistant, WienBot and GeorgiaGov Skill, we were able to explore a new design that still fulfills D115 criteria while only adapting ideas from other interaction models. %  The c
Among the surprising capabilities of Alexa Skills Kit is its ability to deal with large JSON files in a very short time.