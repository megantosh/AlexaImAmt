\chapter{Evaluation}
\label{evaluation}
%DELETEME: The evaluation chapter is one of the most important chapters of your work. Here, you will prove usability/efficiency of your approach by presenting and interpreting your results. You should discuss your results and interprete them, if possible. Drawing conclusions on the results will be one important point that your estimators will refer to when grading your work.



\section{WienBot}

\todo{
-versteht nicht, wenn ich `perso' schreibe\\
-verbose.. "warum möchten Sie aus der Stadt...\\
-uses emojis\\
-speech recognition happens on device (iOS)\\
-use failed attempts of screenshots (new and old)

}
\section{GeorgiaGov}


child care
vs child custody. Sorry i didnt get that. (check first skizze)

\begin{figure}[p]

	\caption[Conversation Flow on Georgia.gov]{Activity Diagram of Conversation Flow on Georgia.gov}
	\centering
	\includegraphics[ height=\textheight]{Georgia2}
	\label{astah:georgiaActivity}
\end{figure}

\todo{
-This is how their model works:\\
services are grouped into categories as seen on \href{http://www.georgia.gov}{georgia.gov}\\
-Alexa catches from any sentence you say only the service grouping name.\\
-then gives you a definition\\
-then asks you if you want to hear about related services (in the format of FAQ)\\
-possible examples: ``I am 17 and my parents are blind. How do I renew my license?''\\
-it goes through this list and you can say yes and no if you want to hear about the `related' item.\\
-of the related items are questions like `how much does it cost', `how much time does it take'. These are no explicitly put in the questions you can ask but they become clear if you have time to make conversation. (Kritikpunkt!)
-then when it reaches the end of the list it says that it's over and if you want to have the phone number that is related tot this service grouping
%
%
%
-then the conversastion either ends or restarts.\\
-you can ask questions only about service groupings\\
%
%
-stop and cancel intent:
\textit{Feel free to ask another question or say exit}
-helpIntent:
Ask me a question
%
repeat does
%
-reprompts are not very ueful and repetitive
}


This way, if a user does not see this service is relevant to them, they will say no and skip to the next

\section{Alexa in General}
Experian of Alexa as a whole \url{https://www.experian.com/innovation/thought-leadership/amazon-echo-consumer-survey.jsp}


\todo{are Skills really going to make it?
	
	30,000 users a day in the US\\
	\url{https://omr.com/en/amazon-alexa-skill-marketing/}
	10,000 BVG

}

\todo{ metrics from Seif's BA!}


\todo{sometimes voice will hardly or never get what you want. try to tell alexa to play YAS. will always redirect you to something else}


\todo{- wo hilft mir alexa, was mach ich lieber woanders?\\
	- wie kann man die G\"ute des Systems beurteilen? \textbf{do not forget the surveyu made}   }

\section{Metrics}

\todo{
\textbf{talk about umfrage-design:}\\
skalen, yes no, choosing between limited answers, etc.\\
combining logic and validation, such that you would be asked about only one thing if you answered the other respectively.
done in english and german for purpse: cater for different people living in berlin\\
conjoint analyse\\
a-b testing - see what users prefer then come to conclusion that dienstleistung is the most important\\
we evaluate how people speak to alexa with sample sentences.\\
ask questions like "Did u know berlin.de has a behindertenberechtigt" to educate the person at the same time\\
}

\todo{
what disturbs you  most about alexa?\\
would u try it again if the question fails?\\
-thats why the retention rate is low
}
\todo{
NPR\\
time series
a-b testing
weighting
use of correlations in the legal system (got, defaulting)
search map predicts ur personality
fear of judging, fear,…interdependence with robots
likelihood of having different social cirlces for a lasting relationship
\\
- see how many use the skill after publishing
tinyurl.com/AlexaBln
\url{https://umfrage.hu-berlin.de/index.php/879458?lang=en}
\url{https://umfrage.hu-berlin.de/index.php/879458?lang=de}
}


\textcolor{magenta}{
-benchmarks\\
-strengths and weaknesses\\
-challenges\\
-performance\\
-usability\\
-feasibility of using the studied agents
\\
- node.js?\\
- amazon's system testing options (incl. Betas)\\
-\\
- system usability scales (ISO, DIN)\\
- Con: Alexa skills are listed in the amazon shop page. Sehr un\"ubersichtlich\\ just like prime\\
- impression: Amazon collects data and makes something "intuitive out of it for you". e.g. fire stick setup already had account linked before connecting to the internet! scary/funny/ but then it could be counterintuitive at some point if u want to do ur own customizations.
\\
- removing bias in recriutment of participants (diversify based on what categories?)\\
-\\
- EVAL: AUC/ROC, true positives, false...no of utterances to text\\
- compare with Wiener Stadportal as a benchmark for a bot\\
https://www.wien.gv.at/bot/
http://www.vienna.at/wienbot-chatbot-der-stadt-wien-informiert-als-virtueller-beamter/5590853
https://digitalcity.wien/wienbot-auszeichnung-fuer-chatbot-der-stadt-wien/
singaporebot
}

%###################################################################################
%###################### Results             ########################################
%###################################################################################
\section{Results}
\label{results}


mention git commits?

\textcolor{red}{
usability metrics:
- heuristic eval - guidelines \textbf{(jakob nielsen, ralf molich whitepaper)}\\
-- biggest usability flaw\\
- cognitive walkthrough\\
-- step-by-step approach\\
-- questions..wil the user tr and achive\\
- pluralistic walkthrough\\
-- panel method\\
- hallway testing\\
- A/B Test\\
- speed and Bottlnecks\\
-\\
- clientele: census / SOEP, who can use the bot\\
- make a small prediction (Bus Analytics)\\
- this Hassloch thing from MKTG
}

%###################################################################################
%###################### Discussions         ########################################
%###################################################################################
\section{Discussions}
\label{discussions}

\todo{will never get things like A\$ap Rocky, Y.A.S elli 3ala spotify not ay wa7da, or Ta3ala (egyptian arabic)}

\todo{you mentioned in intro (structure of this thesis) that you will discuss ML approaches like term weighting}

\todo{
wienbot screenshots	\\
alexa app caught what i said screenshots\\

alexa ask georgia if u cant install it, here is a video
https://vimeo.com/216737044

}

while it may seem trivial at the start to set up, in den details steckt der teufel. from installing npm to ... and the flexibility of the system - so many options to choose from: swagger, dynamo,...



\textcolor{magenta}{
- Evaluate the system:\\
- is it trivial to build such a bot or not / what is the aufwand\\
- how does it react with longer sentences? some service names are long\\
- what does levenstein distanz cause\\
- wie leicht kann ich eine antwort finden auf das was ich suche?\\
- how am i going to classify my tests?\\
-\\
- are chatbots being pushed on the market or is there a demand? (kleine Umfrage basteln?)\\
- how easy or difficult it is to make a bot: planing poker - varianz anschauen zw. leicht und schwer und iterativ darüber sprechen\\
-	wo kann der Kunde (Sawa2 kan el end user or the senat in our case) help optimize the bot
masalan bürgeramt beyektebo, welche Rechtsgrundlage \"keine\"
-	auff\"{a}llige Probleme
masalan zay Perso, PA, personalausweis, how to introduce \"expert mode\" so that if u add it with a special character it knows what u want, just like alexa knows when u rename the lamp - refer again to use cases and exper vs personal field
}

\todo{
mention that building with alexa allows entertainment through small talk and assures the integration of the skill in a widely used platform, but 
small talk gets you out of the skill in the pro/con section. it's not a con but it needs to be addressed. like meaning that u develop an app for iPhone is not enough guarantee that having an iPhone makes people only use ur app.
}
